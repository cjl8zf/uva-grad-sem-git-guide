\documentclass{article}
\usepackage{mystyle}
\usepackage{hyperref}
\usepackage{listings}

\title{Version Controlling Your Thesis With Git\\ {\large University of
  Virginia Graduate Seminar}}
\author{Christopher J.R. Lloyd}

\begin{document}
\maketitle
\section{Installing}
\begin{itemize}
\item For Windows download and install from \href{https://git-scm.com/download/}{https://git-scm.com/download/}
\item On Mac you can check if it is already installed by opening the
  terminal (to open finder open the /Applications/Utilities folder,
  then double-click Terminal) and running:
\begin{lstlisting}[language=bash]
  git --version
\end{lstlisting}
If it is installed it will tell you the version, otherwise it will ask
you if you would like to install it.
\item On Debian-based Linux (including Ubuntu) you can install it via
\begin{lstlisting}[language=bash]
  sudo apt install git-all
\end{lstlisting}
\end{itemize}
\section{Accessing Git}
Once Git has been installed one can access it from the terminal /
command line. On Windows the command line (CMD) is accessed easily by
holding ``Windows Key'' and ``r'' together to open the run menu. Then
type in ``cmd'' and press ``enter''. On Mac the terminal is
opened as explained above. One can check to see if Git is properly
installed by running
\begin{lstlisting}[language=bash]
  git --version
\end{lstlisting}
For me this outputs 
\begin{lstlisting}[language=bash]
  git version 2.24.3
\end{lstlisting}
The way we will interact with Git is by calling commands of the form
\begin{lstlisting}
  git command
\end{lstlisting}
where command is some valid git command. For instance, we have already
seen version is a valid git command. The next useful command is help:
\begin{lstlisting}
  git help
\end{lstlisting}
this will show a list of the valid commands.
\section{Setting Up a Repository}
Suppose you have a LaTeX project folder called \lstinline{test}
containing a LaTeX file \lstinline{test.tex}. To turn this folder into
a git repo, navigate to the folder inside of the terminal using the
\lstinline{cd} command (which stands for change directory), and then
run the command \lstinline{git init}. For me (on Mac) this looks like
\begin{lstlisting}
chris test % git init 
Initialized empty Git repository in /Users/chris/Desktop/test/.git/
chris test % 
\end{lstlisting}
In the notation above everything after \(\%\) is the command I
entered, and before it is the user (chris) followed by the folder
(test). A git repo has been created and so we can start tracking
files. Really this means a hidden folder named \lstinline{.git} has
been created inside of the \lstinline{test} folder where git will keep all of the
necessary data to track our project.
\section{The First Commit}
Making a commit in git is like etching the current state of the
project into stone. It's not that you can't make changes after this,
but there will always be an exact record of the contents of this
folder at the instance of commit. To look
around at what is going on inside of the git repo one uses the
\lstinline{git status} command:
\begin{lstlisting}
chris test % git status
On branch master

No commits yet

Untracked files:
  (use "git add <file>..." to include in what will be committed)
	test.aux
	test.log
	test.pdf
	test.synctex.gz
	test.tex

nothing added to commit but untracked files present (use "git add" to track)
\end{lstlisting}
This tells us what is and what isn't being tracked by the Git
repository. We haven't started tracking anything yet. Notice there are
a bunch of files generated by LaTeX that we really don't care
about. We never want to start tracking these files since they are
auto-generated. We really just want to track the \lstinline{test.tex}
file. We can tell Git to start caring about this file via
\lstinline{git add test.tex} and then running \lstinline{git status}
to see what has changed:
\begin{lstlisting}
chris test % git add test.tex 
chris test % git status
On branch master

No commits yet

Changes to be committed:
  (use "git rm --cached <file>..." to unstage)
	new file:   test.tex

Untracked files:
  (use "git add <file>..." to include in what will be committed)
	test.aux
	test.log
	test.pdf
	test.synctex.gz
\end{lstlisting}
Now we see that the \lstinline{test.tex} is now ready to be
committed. You only ever need to use the \lstinline{git add} command
once to tell the git about the file, after which Git will never
forget.

Now we use the \lstinline{git stage test.tex} command to tell git we
are about to commit this file. This will seem slightly redundant for
our single file example, but this becomes important when one is
committing multiple files that all have a related changes. Next one
runs the \lstinline{git commit -m ``Message''}. The extra option
\lstinline{-m} specifies the message of the commit, which can be
whatever you want. This is useful for when you are looking back at the
changes over time. If one fails to specify the message here, they will
be dropped into the default system text editor, probably vi / vim or emacs...
which could be unpleasant.
\begin{lstlisting}
chris test % git stage test.tex
chris test % git commit -m "First Commit"
[master (root-commit) 88e2b3c] First Commit

 1 file changed, 5 insertions(+)
 create mode 100644 test.tex
chris test % 
\end{lstlisting}
In the above commit my message was ``First Commit''. We can now inspect the
previous commits using \lstinline{git log}:


\begin{lstlisting}
chris % git log
commit 88e2b3cce2f264d3d658785b5bcad876b2c1342a (HEAD -> master)
Author: Chrstopher Lloyd <chris>
Date:   Thu Dec 3 14:26:46 2020 -0500

    First Commit

\end{lstlisting}
This is the most basic usage of git. Now after you make changes to your
file, stage the file, and commit it again with the relevant
message, and this will add to your repo. Notice the long string of
numbers and letters, this is the unique id of this commit. It is
generated via a hashing function of the input (which in itself is
mathematically interesting).

\section{Pushing and Pulling and GitHub}
So far this repo is only stored locally on your computer inside of the
\lstinline{test} folder. We wish to store copies of our repo somewhere
else in case of fire. A good choice is GitHub! After Microsoft bought GitHub any
user can store private repositories for free. After making an account on
\href{https://github.com/}{https://github.com/} you can make a new
repository on the website, set it to private, and maybe name it say
``test''. Once this has been done you can push the local repository
that was just created to github via the commands they present:
\begin{lstlisting}
git remote add origin https://github.com/cjl8zf/test.git
git branch -M master
git push -u origin master
\end{lstlisting}
Note: In Fall 2020 GitHub changed the default branch name of the repos
to main instead of master, but Git still uses master as the
default so here we had to change main to master to be consistent with
the defaults of Git itself.
\begin{lstlisting}
chris test % git remote add origin https://github.com/cjl8zf/test.git 
chris test % git branch -M master
chris test % git push -u origin master 
Enumerating objects: 3, done.
Counting objects: 100% (3/3), done.
Delta compression using up to 12 threads
Compressing objects: 100% (2/2), done.
Writing objects: 100% (3/3), 279 bytes | 279.00 KiB/s, done.
Total 3 (delta 0), reused 0 (delta 0)
To https://github.com/cjl8zf/test.git
 * [new branch]      master -> master
Branch 'master' set up to track remote branch 'master' from 'origin'.
\end{lstlisting}
This will prompt you for your GitHub password. This will only need to
be done one time! Now you should see the tracked file has appeared on GitHub.

The real utility of Git comes from the way it allows multiple authors
/ programmers to collaborate on projects without over-writing other
peoples work. To pull changes from the GitHub you would run
\lstinline{git pull} (only do this after you have committed everything
you are working on currently) this will then download any changes and
incorporate them into your repository. Although, right now there will
be no changes to our \lstinline{test.tex} repo. Suppose we have made
another change, committed it, and then want to add it to the GitHub
repo. We would then use the \lstinline{git push} command. For example
after making a small change in \lstinline{test.tex} we would 
\begin{lstlisting}
chris test % git stage test.tex
chris test % git commit -m "Made a change"
[master 4787a59] Made a change
Committer: Chrstopher Lloyd <chris>

1 file changed, 3 insertions(+), 1 deletion(-)
chris test % git push
Enumerating objects: 5, done.
Counting objects: 100% (5/5), done.
Delta compression using up to 12 threads
Compressing objects: 100% (2/2), done.
Writing objects: 100% (3/3), 330 bytes | 330.00 KiB/s, done.
Total 3 (delta 0), reused 0 (delta 0)
To https://github.com/cjl8zf/test.git
   88e2b3c..4787a59  master -> master
\end{lstlisting}

Now the changes we have made are backed up safely on GitHub!

\section{Cloning a Repo and Grabbing a Thesis Template}
There are many interesting projects on GitHub and due to the
distributed nature of git you can copy the entire project directly
to your computer with one command: \lstinline{git clone}. For us we
will download a copy of a nice UVA thesis template made by Chao Gu
available at
\href{https://github.com/asymmetry/uva-thesis-template}{https://github.com/asymmetry/uva-thesis-template}. Run
the command:
\begin{lstlisting}
git clone https://github.com/asymmetry/uva-thesis-template
Cloning into 'uva-thesis-template'...
remote: Enumerating objects: 30, done.
remote: Total 30 (delta 0), reused 0 (delta 0), pack-reused 30
Unpacking objects: 100% (30/30), done.
\end{lstlisting}
this will download a working thesis template. Note: you do not need a
GitHub account to clone public repositories. You can now edit the
files as you wish and commit to your local copy of this project. Then
you would run
\begin{lstlisting}
git remote remove origin
\end{lstlisting}
so it is no longer connected to the repo you cloned it from. You would
setup a new GitHub project and set that as the origin by following the
GitHub setup steps above.

The source code for these notes can be cloned via:
\begin{lstlisting}
git clone https://github.com/cjl8zf/uva-grad-sem-git-guide
\end{lstlisting}

\section{Overleaf}
There is background integration of Git inside of Overleaf. In the left
sidebar there is the Git option which allows you to clone the Overleaf
project to your computer. You can then push to the origin (which will
already be set up for you)! This will make your changes appear in
Overleaf for everyone else! 
\end{document}


%%% Local Variables:
%%% mode: latex
%%% TeX-master: t
%%% End:
